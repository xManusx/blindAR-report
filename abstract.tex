\begin{abstract}

Visually impaired or even blind people face various challenges in everyday life.
One growing problem of blind people is the rise of touch screens in the last 15 years.
An increasing number of devices are equipped with touch screens nowadays  -- not only smartphones and laptops, but also coffee machines, stoves and even elevators.\\
A short survey among blind people returned, that the increase in touch screen devices is indeed seen as a problem by the blind community and using technological aids to overcome this issue is welcomed.
We used the Microsoft HoloLens as hardware platform to implement \enquote*{blindAR}, a system that allows blind people the use of touch screens with ease.
We implemented  each part of blindAR separately and showed their viability.
Future work will include the complete implementation and a user study to evaluate the usability of the system we envisioned.

%The abstract summarizes your complete project and is the last thing that you write. That means that the reader can find parts of the introduction, methods, results, discussion and outlook in roughly 200 words. After the reader is finished with the abstract he/she must be eager to read the complete article. \\
%The abstract can start with a fact that no one can challenge like ``cats are nice animals''. Then you define a problem in this context like ``However, cats smell''. You go on with introducing your way of solving this problem like ``we used shampoo to solve this problem'' and also mention the most striking result like "we were able to make 89~\% of our test cats odorant''. You finish with an outlook and advertise your work like ``the idea of washing cats can be a first step of significantly improving the relation of cats and humans''. \\
%The abstract is mainly written in simple past, present tense is only used in facts you are absolutely sure about.

\end{abstract}

%\begin{IEEEkeywords}
	%Augmented reality, HoloLens, inclusion, computer vision, usability
%\end{IEEEkeywords}
